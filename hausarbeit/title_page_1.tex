%%%%%%%%%%%%%%%%%%%%%%%%%%%%%%%%%%%%%%%%%
% University Assignment Title Page
% LaTeX Template
% Version 1.0 (27/12/12)
%
% This template has been downloaded from:
% http://www.LaTeXTemplates.com
%
% Original author:
% WikiBooks (http://en.wikibooks.org/wiki/LaTeX/Title_Creation)
%
% License:
% CC BY-NC-SA 3.0 (http://creativecommons.org/licenses/by-nc-sa/3.0/)
%
% Instructions for using this template:
% This title page is capable of being compiled as is. This is not useful for
% including it in another document. To do this, you have two options:
%
% 1) Copy/paste everything between \begin{document} and \end{document}
% starting at \begin{titlepage} and paste this into another LaTeX file where you
% want your title page.
% OR
% 2) Remove everything outside the \begin{titlepage} and \end{titlepage} and
% move this file to the same directory as the LaTeX file you wish to add it to.
% Then add %%%%%%%%%%%%%%%%%%%%%%%%%%%%%%%%%%%%%%%%%
% University Assignment Title Page
% LaTeX Template
% Version 1.0 (27/12/12)
%
% This template has been downloaded from:
% http://www.LaTeXTemplates.com
%
% Original author:
% WikiBooks (http://en.wikibooks.org/wiki/LaTeX/Title_Creation)
%
% License:
% CC BY-NC-SA 3.0 (http://creativecommons.org/licenses/by-nc-sa/3.0/)
%
% Instructions for using this template:
% This title page is capable of being compiled as is. This is not useful for
% including it in another document. To do this, you have two options:
%
% 1) Copy/paste everything between \begin{document} and \end{document}
% starting at \begin{titlepage} and paste this into another LaTeX file where you
% want your title page.
% OR
% 2) Remove everything outside the \begin{titlepage} and \end{titlepage} and
% move this file to the same directory as the LaTeX file you wish to add it to.
% Then add %%%%%%%%%%%%%%%%%%%%%%%%%%%%%%%%%%%%%%%%%
% University Assignment Title Page
% LaTeX Template
% Version 1.0 (27/12/12)
%
% This template has been downloaded from:
% http://www.LaTeXTemplates.com
%
% Original author:
% WikiBooks (http://en.wikibooks.org/wiki/LaTeX/Title_Creation)
%
% License:
% CC BY-NC-SA 3.0 (http://creativecommons.org/licenses/by-nc-sa/3.0/)
%
% Instructions for using this template:
% This title page is capable of being compiled as is. This is not useful for
% including it in another document. To do this, you have two options:
%
% 1) Copy/paste everything between \begin{document} and \end{document}
% starting at \begin{titlepage} and paste this into another LaTeX file where you
% want your title page.
% OR
% 2) Remove everything outside the \begin{titlepage} and \end{titlepage} and
% move this file to the same directory as the LaTeX file you wish to add it to.
% Then add %%%%%%%%%%%%%%%%%%%%%%%%%%%%%%%%%%%%%%%%%
% University Assignment Title Page
% LaTeX Template
% Version 1.0 (27/12/12)
%
% This template has been downloaded from:
% http://www.LaTeXTemplates.com
%
% Original author:
% WikiBooks (http://en.wikibooks.org/wiki/LaTeX/Title_Creation)
%
% License:
% CC BY-NC-SA 3.0 (http://creativecommons.org/licenses/by-nc-sa/3.0/)
%
% Instructions for using this template:
% This title page is capable of being compiled as is. This is not useful for
% including it in another document. To do this, you have two options:
%
% 1) Copy/paste everything between \begin{document} and \end{document}
% starting at \begin{titlepage} and paste this into another LaTeX file where you
% want your title page.
% OR
% 2) Remove everything outside the \begin{titlepage} and \end{titlepage} and
% move this file to the same directory as the LaTeX file you wish to add it to.
% Then add \input{./title_page_1.tex} to your LaTeX file where you want your
% title page.
%
%%%%%%%%%%%%%%%%%%%%%%%%%%%%%%%%%%%%%%%%%

%----------------------------------------------------------------------------------------
%	PACKAGES AND OTHER DOCUMENT CONFIGURATIONS
%----------------------------------------------------------------------------------------

\documentclass[12pt]{article}
\usepackage[utf8]{inputenc}
\usepackage{epigraph}
\usepackage{setspace}

\onehalfspacing
\begin{document}



\begin{titlepage}

\newcommand{\HRule}{\rule{\linewidth}{0.5mm}} % Defines a new command for the horizontal lines, change thickness here

\center % Center everything on the page

%----------------------------------------------------------------------------------------
%	HEADING SECTIONS
%----------------------------------------------------------------------------------------

\textsc{\LARGE Universität Leipzig}\\[1.5cm] % Name of your university/college
\textsc{\Large Hausarbeit}\\[0.5cm] % Major heading such as course name
\textsc{\large zum Seminarvortrag "Free Software Foundation Europe"}\\[0.5cm] % Minor heading such as course title

%----------------------------------------------------------------------------------------
%	TITLE SECTION
%----------------------------------------------------------------------------------------

\HRule \\[0.4cm]
{ \huge \bfseries Free Software Foundation Europe und Lobbying}\\[0.4cm] % Title of your document
\HRule \\[1.5cm]

%----------------------------------------------------------------------------------------
%	AUTHOR SECTION
%----------------------------------------------------------------------------------------

\begin{minipage}{0.4\textwidth}
\begin{flushleft} \large
\emph{Author:}\\
Sascha \textsc{Ebert} % Your name
\end{flushleft}
\end{minipage}
~
\begin{minipage}{0.4\textwidth}
\begin{flushright} \large
\emph{Supervisor:} \\
Prof. Dr. H.-G. \textsc{Gräbe} % Supervisor's Name
\end{flushright}
\end{minipage}\\[4cm]

% If you don't want a supervisor, uncomment the two lines below and remove the section above
%\Large \emph{Author:}\\
%John \textsc{Smith}\\[3cm] % Your name

%----------------------------------------------------------------------------------------
%	DATE SECTION
%----------------------------------------------------------------------------------------

{\large \today}\\[3cm] % Date, change the \today to a set date if you want to be precise

%----------------------------------------------------------------------------------------
%	LOGO SECTION
%----------------------------------------------------------------------------------------

%\includegraphics{Logo}\\[1cm] % Include a department/university logo - this will require the graphicx package

%----------------------------------------------------------------------------------------

\vfill % Fill the rest of the page with whitespace

\end{titlepage}

\section{Einleitung}
\epigraph{We believe in cooperation and transparency.}{FSFE}
Die öffentlichen Worte der Free Software Foundation Europe (FSFE) sind deutlich. Es geht
um das Verwirklichen ihrer Kernziele mit Hilfe eines transparenten Aktionsmusters.
Dieses Statement überrascht kaum, da sich die FSFE die Erhaltung der Vision von ``Free Software'' zum Ziel gemacht hat, dessen Definition gerade die Gedanken ``Transparenz'' und ``Korporation'' als absolute Kernfaktoren enthält. Es ergibt sich somit direkt folgend die Forderung, die Folgen der Definition des Begriffes "Free Software" konsistent in ihre Struktur und Organisation zu übertragen.

Die Frage der Finanzierung der FSFE und deren Arbeit in diversen
Kampagnen bildet nun das Gerüst für einen intuitiven Lobbyismusvorwurf. Dieser muss klar ausgeführt werden und mit Hilfe
verschiedener Werke auf die ausgiebige Reichweite und Komplexität
des Begriffs angewendet werden.

Diese Arbeit schickt sich an die Gestalt und Struktur der Free Software Foundation Europe zu erläutern, den eben genannten Vorwurf zu spezifizieren und theoretisch zu behandeln inwiefern und in welchen Formen die FSFE Lobbying betreibt. Ergänzend soll ein Vergleich gegenüber anderen Lobbyismusstrukturen,
wie die der Pharmaindustrie geführt werden, um auf die spezielle Verbindung zwischen dem Thema der FSFE und dem Lobbyismus aufmerksam zu machen.

\newpage
\section{Free Software Foundation Europe}
\epigraph{Free Software Foundation Europe is a charity that empowers users to control technology.}{FSFE}
Seit ihrer Gründung am 10. März 2001 hat die FSFE einen Langen Weg der Vertretung Freier Software bestritten. Nach Gerog Greve, Karsten Gerloff ist Matthias Kirschner nunmehr der bereits 3. Präsident der FSFE.

\begin{itemize}
\setlength\itemsep{0em}
\item Einsteigen über George Greve Interview, Kampf gegen die Bedrohungen für freie Software, klare Ziele
\item Self Conception (Zitat von Website, stichworte hervorheben) (Norbert)
\item WTF! Free Software - Definition (Norbert)
\item FS vs. Open Source (Sascha)
\item Lizensen (mindmap mit zusammenhängen) (Paul)
\item Kampagnen (Logos) n viele (Norbert)
\item Aufbau des Vereins (Hierarchie) (Paul)
\item Fellowships (Sascha)
\item Finanzierung I (Einnahmen Spenden) (Paul)
\item Finanzierung II (Einnahmen nach Firmen) (Paul)
\item welchen Einfluss hat die FSFE Überhaupt?
\item welche Unternehmen haben wie Einfluss auf die FSFE?
\end{itemize}

\section{Lobbyismusbegriff}

\subsection{Begriff}

\paragraph{Notizen}
\begin{itemize}
\item verschiedene Begriffe erläutern
\item Vergleichsprinzip erläutern
\item Lobbyvorwurf durch Anwendung der Definition spezifizieren
\end{itemize}

\section{Unterschied zu anderen Lobbyingarten}

\paragraph{Notizen}
\begin{itemize}
\item Pharmalobby
\item Agrar-Lobbying
\end{itemize}

\section{Unterschied zu anderen Lobbyingarten}

\paragraph{Notizen}
\begin{itemize}
\item Lobbying als natürlicher Vorgang (Eigene These)
\end{itemize}

\cite{LeifSpeth200312}


\newpage
\bibliography{refs}
\bibliographystyle{ieeetr}

\end{document}
 to your LaTeX file where you want your
% title page.
%
%%%%%%%%%%%%%%%%%%%%%%%%%%%%%%%%%%%%%%%%%

%----------------------------------------------------------------------------------------
%	PACKAGES AND OTHER DOCUMENT CONFIGURATIONS
%----------------------------------------------------------------------------------------

\documentclass[12pt]{article}
\usepackage[utf8]{inputenc}
\usepackage{epigraph}
\usepackage{setspace}

\onehalfspacing
\begin{document}



\begin{titlepage}

\newcommand{\HRule}{\rule{\linewidth}{0.5mm}} % Defines a new command for the horizontal lines, change thickness here

\center % Center everything on the page

%----------------------------------------------------------------------------------------
%	HEADING SECTIONS
%----------------------------------------------------------------------------------------

\textsc{\LARGE Universität Leipzig}\\[1.5cm] % Name of your university/college
\textsc{\Large Hausarbeit}\\[0.5cm] % Major heading such as course name
\textsc{\large zum Seminarvortrag "Free Software Foundation Europe"}\\[0.5cm] % Minor heading such as course title

%----------------------------------------------------------------------------------------
%	TITLE SECTION
%----------------------------------------------------------------------------------------

\HRule \\[0.4cm]
{ \huge \bfseries Free Software Foundation Europe und Lobbying}\\[0.4cm] % Title of your document
\HRule \\[1.5cm]

%----------------------------------------------------------------------------------------
%	AUTHOR SECTION
%----------------------------------------------------------------------------------------

\begin{minipage}{0.4\textwidth}
\begin{flushleft} \large
\emph{Author:}\\
Sascha \textsc{Ebert} % Your name
\end{flushleft}
\end{minipage}
~
\begin{minipage}{0.4\textwidth}
\begin{flushright} \large
\emph{Supervisor:} \\
Prof. Dr. H.-G. \textsc{Gräbe} % Supervisor's Name
\end{flushright}
\end{minipage}\\[4cm]

% If you don't want a supervisor, uncomment the two lines below and remove the section above
%\Large \emph{Author:}\\
%John \textsc{Smith}\\[3cm] % Your name

%----------------------------------------------------------------------------------------
%	DATE SECTION
%----------------------------------------------------------------------------------------

{\large \today}\\[3cm] % Date, change the \today to a set date if you want to be precise

%----------------------------------------------------------------------------------------
%	LOGO SECTION
%----------------------------------------------------------------------------------------

%\includegraphics{Logo}\\[1cm] % Include a department/university logo - this will require the graphicx package

%----------------------------------------------------------------------------------------

\vfill % Fill the rest of the page with whitespace

\end{titlepage}

\section{Einleitung}
\epigraph{We believe in cooperation and transparency.}{FSFE}
Die öffentlichen Worte der Free Software Foundation Europe (FSFE) sind deutlich. Es geht
um das Verwirklichen ihrer Kernziele mit Hilfe eines transparenten Aktionsmusters.
Dieses Statement überrascht kaum, da sich die FSFE die Erhaltung der Vision von ``Free Software'' zum Ziel gemacht hat, dessen Definition gerade die Gedanken ``Transparenz'' und ``Korporation'' als absolute Kernfaktoren enthält. Es ergibt sich somit direkt folgend die Forderung, die Folgen der Definition des Begriffes "Free Software" konsistent in ihre Struktur und Organisation zu übertragen.

Die Frage der Finanzierung der FSFE und deren Arbeit in diversen
Kampagnen bildet nun das Gerüst für einen intuitiven Lobbyismusvorwurf. Dieser muss klar ausgeführt werden und mit Hilfe
verschiedener Werke auf die ausgiebige Reichweite und Komplexität
des Begriffs angewendet werden.

Diese Arbeit schickt sich an die Gestalt und Struktur der Free Software Foundation Europe zu erläutern, den eben genannten Vorwurf zu spezifizieren und theoretisch zu behandeln inwiefern und in welchen Formen die FSFE Lobbying betreibt. Ergänzend soll ein Vergleich gegenüber anderen Lobbyismusstrukturen,
wie die der Pharmaindustrie geführt werden, um auf die spezielle Verbindung zwischen dem Thema der FSFE und dem Lobbyismus aufmerksam zu machen.

\newpage
\section{Free Software Foundation Europe}
\epigraph{Free Software Foundation Europe is a charity that empowers users to control technology.}{FSFE}
Seit ihrer Gründung am 10. März 2001 hat die FSFE einen Langen Weg der Vertretung Freier Software bestritten. Nach Gerog Greve, Karsten Gerloff ist Matthias Kirschner nunmehr der bereits 3. Präsident der FSFE.

\begin{itemize}
\setlength\itemsep{0em}
\item Einsteigen über George Greve Interview, Kampf gegen die Bedrohungen für freie Software, klare Ziele
\item Self Conception (Zitat von Website, stichworte hervorheben) (Norbert)
\item WTF! Free Software - Definition (Norbert)
\item FS vs. Open Source (Sascha)
\item Lizensen (mindmap mit zusammenhängen) (Paul)
\item Kampagnen (Logos) n viele (Norbert)
\item Aufbau des Vereins (Hierarchie) (Paul)
\item Fellowships (Sascha)
\item Finanzierung I (Einnahmen Spenden) (Paul)
\item Finanzierung II (Einnahmen nach Firmen) (Paul)
\item welchen Einfluss hat die FSFE Überhaupt?
\item welche Unternehmen haben wie Einfluss auf die FSFE?
\end{itemize}

\section{Lobbyismusbegriff}

\subsection{Begriff}

\paragraph{Notizen}
\begin{itemize}
\item verschiedene Begriffe erläutern
\item Vergleichsprinzip erläutern
\item Lobbyvorwurf durch Anwendung der Definition spezifizieren
\end{itemize}

\section{Unterschied zu anderen Lobbyingarten}

\paragraph{Notizen}
\begin{itemize}
\item Pharmalobby
\item Agrar-Lobbying
\end{itemize}

\section{Unterschied zu anderen Lobbyingarten}

\paragraph{Notizen}
\begin{itemize}
\item Lobbying als natürlicher Vorgang (Eigene These)
\end{itemize}

\cite{LeifSpeth200312}


\newpage
\bibliography{refs}
\bibliographystyle{ieeetr}

\end{document}
 to your LaTeX file where you want your
% title page.
%
%%%%%%%%%%%%%%%%%%%%%%%%%%%%%%%%%%%%%%%%%

%----------------------------------------------------------------------------------------
%	PACKAGES AND OTHER DOCUMENT CONFIGURATIONS
%----------------------------------------------------------------------------------------

\documentclass[12pt]{article}
\usepackage[utf8]{inputenc}
\usepackage{epigraph}
\usepackage{setspace}

\onehalfspacing
\begin{document}



\begin{titlepage}

\newcommand{\HRule}{\rule{\linewidth}{0.5mm}} % Defines a new command for the horizontal lines, change thickness here

\center % Center everything on the page

%----------------------------------------------------------------------------------------
%	HEADING SECTIONS
%----------------------------------------------------------------------------------------

\textsc{\LARGE Universität Leipzig}\\[1.5cm] % Name of your university/college
\textsc{\Large Hausarbeit}\\[0.5cm] % Major heading such as course name
\textsc{\large zum Seminarvortrag "Free Software Foundation Europe"}\\[0.5cm] % Minor heading such as course title

%----------------------------------------------------------------------------------------
%	TITLE SECTION
%----------------------------------------------------------------------------------------

\HRule \\[0.4cm]
{ \huge \bfseries Free Software Foundation Europe und Lobbying}\\[0.4cm] % Title of your document
\HRule \\[1.5cm]

%----------------------------------------------------------------------------------------
%	AUTHOR SECTION
%----------------------------------------------------------------------------------------

\begin{minipage}{0.4\textwidth}
\begin{flushleft} \large
\emph{Author:}\\
Sascha \textsc{Ebert} % Your name
\end{flushleft}
\end{minipage}
~
\begin{minipage}{0.4\textwidth}
\begin{flushright} \large
\emph{Supervisor:} \\
Prof. Dr. H.-G. \textsc{Gräbe} % Supervisor's Name
\end{flushright}
\end{minipage}\\[4cm]

% If you don't want a supervisor, uncomment the two lines below and remove the section above
%\Large \emph{Author:}\\
%John \textsc{Smith}\\[3cm] % Your name

%----------------------------------------------------------------------------------------
%	DATE SECTION
%----------------------------------------------------------------------------------------

{\large \today}\\[3cm] % Date, change the \today to a set date if you want to be precise

%----------------------------------------------------------------------------------------
%	LOGO SECTION
%----------------------------------------------------------------------------------------

%\includegraphics{Logo}\\[1cm] % Include a department/university logo - this will require the graphicx package

%----------------------------------------------------------------------------------------

\vfill % Fill the rest of the page with whitespace

\end{titlepage}

\section{Einleitung}
\epigraph{We believe in cooperation and transparency.}{FSFE}
Die öffentlichen Worte der Free Software Foundation Europe (FSFE) sind deutlich. Es geht
um das Verwirklichen ihrer Kernziele mit Hilfe eines transparenten Aktionsmusters.
Dieses Statement überrascht kaum, da sich die FSFE die Erhaltung der Vision von ``Free Software'' zum Ziel gemacht hat, dessen Definition gerade die Gedanken ``Transparenz'' und ``Korporation'' als absolute Kernfaktoren enthält. Es ergibt sich somit direkt folgend die Forderung, die Folgen der Definition des Begriffes "Free Software" konsistent in ihre Struktur und Organisation zu übertragen.

Die Frage der Finanzierung der FSFE und deren Arbeit in diversen
Kampagnen bildet nun das Gerüst für einen intuitiven Lobbyismusvorwurf. Dieser muss klar ausgeführt werden und mit Hilfe
verschiedener Werke auf die ausgiebige Reichweite und Komplexität
des Begriffs angewendet werden.

Diese Arbeit schickt sich an die Gestalt und Struktur der Free Software Foundation Europe zu erläutern, den eben genannten Vorwurf zu spezifizieren und theoretisch zu behandeln inwiefern und in welchen Formen die FSFE Lobbying betreibt. Ergänzend soll ein Vergleich gegenüber anderen Lobbyismusstrukturen,
wie die der Pharmaindustrie geführt werden, um auf die spezielle Verbindung zwischen dem Thema der FSFE und dem Lobbyismus aufmerksam zu machen.

\newpage
\section{Free Software Foundation Europe}
\epigraph{Free Software Foundation Europe is a charity that empowers users to control technology.}{FSFE}
Seit ihrer Gründung am 10. März 2001 hat die FSFE einen Langen Weg der Vertretung Freier Software bestritten. Nach Gerog Greve, Karsten Gerloff ist Matthias Kirschner nunmehr der bereits 3. Präsident der FSFE.

\begin{itemize}
\setlength\itemsep{0em}
\item Einsteigen über George Greve Interview, Kampf gegen die Bedrohungen für freie Software, klare Ziele
\item Self Conception (Zitat von Website, stichworte hervorheben) (Norbert)
\item WTF! Free Software - Definition (Norbert)
\item FS vs. Open Source (Sascha)
\item Lizensen (mindmap mit zusammenhängen) (Paul)
\item Kampagnen (Logos) n viele (Norbert)
\item Aufbau des Vereins (Hierarchie) (Paul)
\item Fellowships (Sascha)
\item Finanzierung I (Einnahmen Spenden) (Paul)
\item Finanzierung II (Einnahmen nach Firmen) (Paul)
\item welchen Einfluss hat die FSFE Überhaupt?
\item welche Unternehmen haben wie Einfluss auf die FSFE?
\end{itemize}

\section{Lobbyismusbegriff}

\subsection{Begriff}

\paragraph{Notizen}
\begin{itemize}
\item verschiedene Begriffe erläutern
\item Vergleichsprinzip erläutern
\item Lobbyvorwurf durch Anwendung der Definition spezifizieren
\end{itemize}

\section{Unterschied zu anderen Lobbyingarten}

\paragraph{Notizen}
\begin{itemize}
\item Pharmalobby
\item Agrar-Lobbying
\end{itemize}

\section{Unterschied zu anderen Lobbyingarten}

\paragraph{Notizen}
\begin{itemize}
\item Lobbying als natürlicher Vorgang (Eigene These)
\end{itemize}

\cite{LeifSpeth200312}


\newpage
\bibliography{refs}
\bibliographystyle{ieeetr}

\end{document}
 to your LaTeX file where you want your
% title page.
%
%%%%%%%%%%%%%%%%%%%%%%%%%%%%%%%%%%%%%%%%%

%----------------------------------------------------------------------------------------
%	PACKAGES AND OTHER DOCUMENT CONFIGURATIONS
%----------------------------------------------------------------------------------------

\documentclass[12pt]{article}
\usepackage[utf8]{inputenc}
\usepackage{epigraph}
\usepackage{setspace}

\onehalfspacing
\begin{document}



\begin{titlepage}

\newcommand{\HRule}{\rule{\linewidth}{0.5mm}} % Defines a new command for the horizontal lines, change thickness here

\center % Center everything on the page

%----------------------------------------------------------------------------------------
%	HEADING SECTIONS
%----------------------------------------------------------------------------------------

\textsc{\LARGE Universität Leipzig}\\[1.5cm] % Name of your university/college
\textsc{\Large Hausarbeit}\\[0.5cm] % Major heading such as course name
\textsc{\large zum Seminarvortrag "Free Software Foundation Europe"}\\[0.5cm] % Minor heading such as course title

%----------------------------------------------------------------------------------------
%	TITLE SECTION
%----------------------------------------------------------------------------------------

\HRule \\[0.4cm]
{ \huge \bfseries Free Software Foundation Europe und Lobbying}\\[0.4cm] % Title of your document
\HRule \\[1.5cm]

%----------------------------------------------------------------------------------------
%	AUTHOR SECTION
%----------------------------------------------------------------------------------------

\begin{minipage}{0.4\textwidth}
\begin{flushleft} \large
\emph{Author:}\\
Sascha \textsc{Ebert} % Your name
\end{flushleft}
\end{minipage}
~
\begin{minipage}{0.4\textwidth}
\begin{flushright} \large
\emph{Supervisor:} \\
Prof. Dr. H.-G. \textsc{Gräbe} % Supervisor's Name
\end{flushright}
\end{minipage}\\[4cm]

% If you don't want a supervisor, uncomment the two lines below and remove the section above
%\Large \emph{Author:}\\
%John \textsc{Smith}\\[3cm] % Your name

%----------------------------------------------------------------------------------------
%	DATE SECTION
%----------------------------------------------------------------------------------------

{\large \today}\\[3cm] % Date, change the \today to a set date if you want to be precise

%----------------------------------------------------------------------------------------
%	LOGO SECTION
%----------------------------------------------------------------------------------------

%\includegraphics{Logo}\\[1cm] % Include a department/university logo - this will require the graphicx package

%----------------------------------------------------------------------------------------

\vfill % Fill the rest of the page with whitespace

\end{titlepage}

\section{Einleitung}
\epigraph{We believe in cooperation and transparency.}{FSFE}
Die öffentlichen Worte der Free Software Foundation Europe (FSFE) sind deutlich. Es geht
um das Verwirklichen ihrer Kernziele mit Hilfe eines transparenten Aktionsmusters.
Dieses Statement überrascht kaum, da sich die FSFE die Erhaltung der Vision von ``Free Software'' zum Ziel gemacht hat, dessen Definition gerade die Gedanken ``Transparenz'' und ``Korporation'' als absolute Kernfaktoren enthält. Es ergibt sich somit direkt folgend die Forderung, die Folgen der Definition des Begriffes "Free Software" konsistent in ihre Struktur und Organisation zu übertragen.

Die Frage der Finanzierung der FSFE und deren Arbeit in diversen
Kampagnen bildet nun das Gerüst für einen intuitiven Lobbyismusvorwurf. Dieser muss klar ausgeführt werden und mit Hilfe
verschiedener Werke auf die ausgiebige Reichweite und Komplexität
des Begriffs angewendet werden.

Diese Arbeit schickt sich an die Gestalt und Struktur der Free Software Foundation Europe zu erläutern, den eben genannten Vorwurf zu spezifizieren und theoretisch zu behandeln inwiefern und in welchen Formen die FSFE Lobbying betreibt. Ergänzend soll ein Vergleich gegenüber anderen Lobbyismusstrukturen,
wie die der Pharmaindustrie geführt werden, um auf die spezielle Verbindung zwischen dem Thema der FSFE und dem Lobbyismus aufmerksam zu machen.

\newpage
\section{Free Software Foundation Europe}
\epigraph{Free Software Foundation Europe is a charity that empowers users to control technology.}{FSFE}
Seit ihrer Gründung am 10. März 2001 hat die FSFE einen Langen Weg der Vertretung Freier Software bestritten. Nach Gerog Greve, Karsten Gerloff ist Matthias Kirschner nunmehr der bereits 3. Präsident der FSFE.

\begin{itemize}
\setlength\itemsep{0em}
\item Einsteigen über George Greve Interview, Kampf gegen die Bedrohungen für freie Software, klare Ziele
\item Self Conception (Zitat von Website, stichworte hervorheben) (Norbert)
\item WTF! Free Software - Definition (Norbert)
\item FS vs. Open Source (Sascha)
\item Lizensen (mindmap mit zusammenhängen) (Paul)
\item Kampagnen (Logos) n viele (Norbert)
\item Aufbau des Vereins (Hierarchie) (Paul)
\item Fellowships (Sascha)
\item Finanzierung I (Einnahmen Spenden) (Paul)
\item Finanzierung II (Einnahmen nach Firmen) (Paul)
\item welchen Einfluss hat die FSFE Überhaupt?
\item welche Unternehmen haben wie Einfluss auf die FSFE?
\end{itemize}

\section{Lobbyismusbegriff}

\subsection{Begriff}

\paragraph{Notizen}
\begin{itemize}
\item verschiedene Begriffe erläutern
\item Vergleichsprinzip erläutern
\item Lobbyvorwurf durch Anwendung der Definition spezifizieren
\end{itemize}

\section{Unterschied zu anderen Lobbyingarten}

\paragraph{Notizen}
\begin{itemize}
\item Pharmalobby
\item Agrar-Lobbying
\end{itemize}

\section{Unterschied zu anderen Lobbyingarten}

\paragraph{Notizen}
\begin{itemize}
\item Lobbying als natürlicher Vorgang (Eigene These)
\end{itemize}

\cite{LeifSpeth200312}


\newpage
\bibliography{refs}
\bibliographystyle{ieeetr}

\end{document}
